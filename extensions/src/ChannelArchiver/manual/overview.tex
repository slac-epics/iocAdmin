\chapter{Overview}
The Channel Archiver is an archiving toolset for the Experimental Physics
and Industrial Control System, \INDEX{EPICS}~\cite{ANLweb}.
It can archive any value that is available via \INDEX{ChannelAccess}
(\INDEX{CA}), the EPICS network protocol~\cite{hill89}.
We use the term ``archiver'' whenever we refer to the collection of
programs which allow us to take samples, place them into some storage
and retrieve them again.

\section{Audience}
Casual users will probably only need to know how to run the ``Java Archive
Viewer'' program and read its online help ---
none of which is part of this manual, except for a brief glimpse in 
section \ref{sec:javaclient}.
They may refer to the background information in chapter
\ref{ch:background} for a general understanding of the archiver.

Engineers who configure archive engines will need to be familiar with
the fundamentals up to and including the ArchiveEngine description in
chapter \ref{sec:engine}. They also need to understand the Example
Setup, chapter \ref{ch:examplesetup}, unless their site uses a
different setup which is then described in site-specific
documentation.

If you are stuck with installing and maintaining the archiver at your
site, you have to read and memorize this full document. Sorry. The table of
contents and index are meant to help, but you probably have to read it
once cover to cover.

\section{Channel Archiver Toolset}
The archiver toolset roughly splits into the following pieces:

\begin{description}
\item[\sffamily Sampling:]
The ArchiveEngine collects data from a given list of ChannelAccess
Channels.  The details of when a sample is taken etc.\ can be
configured: One may store every change, store changes that exceed a
dead band (that is configured on the CA server) or use periodic
scanning.
The configuration and operation of the ArchiveEngines will obviously
require some planning, as only data that was sampled and stored will
be available for future retrieval and analysis. Some sensible
compromise will have to be made between the urge to store all
miniscule changes of all the available channels at a site on one hand,
and data storage constraints on the other.

\item[\sffamily Storage:]
The data is stored in binary index and data files. Most end users need not
be concerned about the internals of those files, not even where they
are located, because additional indices allow several sub-archives to
appear like one, bigger, combined archive.
Somebody at each site, though, will need to perform maintenance
tasks: Decide where the data sets are located, how they are
backed up and how users can access them. 

\item[\sffamily Retrieval:]
The archiver toolset provides generic retrieval tools for browsing the
available channels and values, including simple multi-channel
comparisons.
An API allows users to write more sophisticated data analysis tools,
including an XML-RPC based network protocol for remote clients.
\end{description}

